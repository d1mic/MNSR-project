% !TEX encoding = UTF-8 Unicode
\documentclass[a4paper]{article}

\usepackage{color}
\usepackage{url}
\usepackage[T2A]{fontenc} % enable Cyrillic fonts
\usepackage[utf8]{inputenc} % make weird characters work
\usepackage{graphicx}

\usepackage[english,serbian]{babel}
%\usepackage[english,serbianc]{babel} %ukljuciti babel sa ovim opcijama, umesto gornjim, ukoliko se koristi cirilica

\usepackage[unicode]{hyperref}
\hypersetup{colorlinks,citecolor=green,filecolor=green,linkcolor=blue,urlcolor=blue}

\usepackage{listings}

%\newtheorem{primer}{Пример}[section] %ćirilični primer
\newtheorem{primer}{Primer}[section]

\definecolor{mygreen}{rgb}{0,0.6,0}
\definecolor{mygray}{rgb}{0.5,0.5,0.5}
\definecolor{mymauve}{rgb}{0.58,0,0.82}

\lstset{ 
  backgroundcolor=\color{white},   % choose the background color; you must add \usepackage{color} or \usepackage{xcolor}; should come as last argument
  basicstyle=\scriptsize\ttfamily,        % the size of the fonts that are used for the code
  breakatwhitespace=false,         % sets if automatic breaks should only happen at whitespace
  breaklines=true,                 % sets automatic line breaking
  captionpos=b,                    % sets the caption-position to bottom
  commentstyle=\color{mygreen},    % comment style
  deletekeywords={...},            % if you want to delete keywords from the given language
  escapeinside={\%*}{*)},          % if you want to add LaTeX within your code
  extendedchars=true,              % lets you use non-ASCII characters; for 8-bits encodings only, does not work with UTF-8
  firstnumber=1000,                % start line enumeration with line 1000
  frame=single,	                   % adds a frame around the code
  keepspaces=true,                 % keeps spaces in text, useful for keeping indentation of code (possibly needs columns=flexible)
  keywordstyle=\color{blue},       % keyword style
  language=Python,                 % the language of the code
  morekeywords={*,...},            % if you want to add more keywords to the set
  numbers=left,                    % where to put the line-numbers; possible values are (none, left, right)
  numbersep=5pt,                   % how far the line-numbers are from the code
  numberstyle=\tiny\color{mygray}, % the style that is used for the line-numbers
  rulecolor=\color{black},         % if not set, the frame-color may be changed on line-breaks within not-black text (e.g. comments (green here))
  showspaces=false,                % show spaces everywhere adding particular underscores; it overrides 'showstringspaces'
  showstringspaces=false,          % underline spaces within strings only
  showtabs=false,                  % show tabs within strings adding particular underscores
  stepnumber=2,                    % the step between two line-numbers. If it's 1, each line will be numbered
  stringstyle=\color{mymauve},     % string literal style
  tabsize=2,	                   % sets default tabsize to 2 spaces
  title=\lstname                   % show the filename of files included with \lstinputlisting; also try caption instead of title
}

%Page break after paragraph title 
\makeatletter
\renewcommand\paragraph{\@startsection{paragraph}{4}{\z@}%
   {-3.25ex\@plus -1ex \@minus -.2ex}%
   {1.5ex \@plus .2ex}%
   {\normalfont\normalsize\bfseries}}
\makeatother

\begin{document}

\title{Hakovanje i Malware\\ \small{Seminarski rad u okviru kursa\\Metodologija stručnog i naučnog rada\\ Matematički fakultet}}

\author{Nikola Dimić, Đorđe Pantelić, Nikola Živković,\\Mladen Canović\\ dimic.nikola@gmail.com, pantelicdjole94@gmail.com,\\nmzivkovic@gmail.com, mladen.canovic@gmail.com}

%\date{9.~april 2015.}

\maketitle

\abstract{
OVO NE SMEMO ZABORAVITI DA NAPISEMO !!!!!!!!! SLAZEM SE!!!

U ovom tekstu je ukratko prikazana osnovna forma seminarskog rada. Obratite pažnju da je pored ove .pdf datoteke, u prilogu i odgovarajuća .tex datoteka, kao i .bib datoteka korišćena za generisanje literature. Na prvoj strani seminarskog rada su naslov, apstrakt i sadržaj, i to sve mora da stane na prvu stranu! Kako bi Vaš seminarski zadovoljio standarde i očekivanja, koristite uputstva i materijale sa predavanja na temu pisanja seminarskih radova. Ovo je samo šablon koji se odnosi na fizički izgled seminarskog rada (šablon koji \emph{morate} da koristite!) kao i par tehničkih pomoćnih uputstava. Pročitajte tekst pažljivo jer on sadrži i važne informacije vezane za zahteve obima i karakteristika seminarskog rada.}

\setcounter{tocdepth}{2}
\tableofcontents

\newpage

\section{Uvod}
\label{sec:uvod}
U originalnom značenju reč haker je predstavljala osobu koja je, na inovativan način, pokušavala da modifikuje sistem tako da on radi nešto novo. Hakerska kultura nastala je na MIT-u tokom 50-ih i 60-ih godina prošlog veka. Za njih “hack” nije predstavljao samo novi uređaj ili dodavanje funkcionalnosti uređajima, već i prikaz njihove virtuoznosti. Nazvati nekoga hakerom, bio je izraz poštovanja.

Nakon pohađanja kursa o osnovama računarstva 1959. godine, pojedini hakeri su prebacili fokus svojih interesovanja na računare \cite{hackers_heroes}. Izraz “haker”, sada se više odnosio na “osobu, koja uživa u potpunom razumevanju internog funkcionisanja sistema, računara i mreža” \cite{hacker_definition}.

\section{Hakovanje}
Hakovanje (eng.~{\em Hacking}) može predstavljati iskorišćavanje nedostataka sistema, računara ili računarskih mreža. Najčešća meta jesu ljudi koji rade na tom sistemu i ova vrsta napada predstavlja društveni inženjering (eng. \textit{social engineering}).

Pored zlonamernih napadača (eng. \textit{Black hat}), hakeri uz dozvolu administratora upadaju u sisteme u cilju testiranja njihove sigurnosti. Oni predstavljaju etičke hakere (eng. \textit{White hat}). Hakeri iz sive zone (eng. \textit{Gray hat}), zarad dokazivanja svojih veština, vrše napade bez dozvole ali ne u nameri da nanesu značajnu štetu. Pored njih postoje hakeri koji se žalažu za određenu ideju ili cilj (eng. \textit{Hacktivists}), čiji su najpoznatiji predstavnici grupa Anonimusi (eng. \textit{Anonymous}) \cite{hackers_hat}.
 
\subsection{Umetnost upada}
\label{sec:intrusion}

Sa razvojem informacionih tehnologija i računarstva rasla je i pretnja od njihove zloupotrebe. 

Za hakovanje su interesovanje prvobitno pokazale mlade osobe, koje su se time bavile zbog ličnog uzbuđenja \ref{mladi_hakeri} ili grupe profesionalaca koje bi koristile svoju ekspertizu zarad materijalne dobiti \ref{milionska_prevara_kazina}.

Danas države ulažu ogromna sredstva kako u bezbednost, tako i u regrutovanje talentovanih pojedinaca u političke svrhe i napade na vojne ciljeve \ref{cyberwarfare} \cite{intrusion}.

\subsubsection{Milionska prevara kazina}
\label{milionska_prevara_kazina}

Ranih devedesetih godina, Aleks Mejfild je sa tri prijatelja prezentovao softversko rešenje na konferenciji u Las Vegasu. Nakon posete kockarnici, došli su na ideju hakovanja poker aparata. Poker aparati su imali određene hardverske propuste - postojala je mogućnost zamene originalnog čipa. Zarad testiranja svojih programerskih sposobnosti oni su hakovali aparat na drugačiji način.

Napadači su kupili stariji aparat, jer ranije bezbednosti nije poklanjana velika pažnja, kako bi analizirali njegov algoritam. Koristeći disambler, koji su sami razvili, preveli su izvršni kod u izvorni. Analizom ključnih sekcija pronašli su generator nasumičnih brojeva iz šezdesetih godina, koji je predstavljao listu od oko četiri milijarde uskladištenih brojeva kroz koju se iteriralo na svaki deseti deo sekunde.

Po izvlačenju pet karata izvlačilo se još pet, koje bi bile rezerve u slučaju da igrač želi da zameni neke od prvobitno izvučenih. Tih deset karata predstavlja deset uzastopnih brojeva iz generatora. Pametnim rasporedom brojeva u početnoj listi, aparat kontroliše koja je tačno verovatnoća da igrač pobedi. 

Napadači su uz pomoć štoperice i programa, koji su razvili, znali tačno kada će aparat izbaciti dobitnu kombinaciju. Program je na osnovu prvobitno izvučenih karata mogao da odredi trenutno stanje generatora.
Nakon što se njihovo rešenje pokazalo kao uspešno, kupili su moderniji aparat kako bi povećali broj mesta na kojima mogu primeniti svoj metod. Ovaj aparat je imao dva generatora nasumičnih brojeva. Umesto da se svaka izvučena karta određuje na osnovu zbira generatora, iz istog razloga kao u prethodnom slučaju, za sve karte iz jednog izvlačenja korišćena je konstatna vrednost iz drugog generatora. Problem je ovim sveden na problem kriptografije gde su vrednosti prvog generatora pomerene za određeni korak koji se u svakom izvlačenju menja.

Za dve godine oštetili su kazino za preko milion dolara. Vremenom su napadači postali sve manje oprezni što je dovelo do otkrivanja njihove prevare. U zamenu za informacije o bezbednosnim propustima nisu procesuirani \cite{intrusion}.

\subsubsection{Informatičko ratovanje}
\label{cyberwarfare}

Informatičko ratovanje (eng. \textit{Cyberwarfare}) predstavlja način na koje države, bez stavljanja ljudskih života u opasnost, ostvaruju svoje ciljeve.

\textit{Stuxnet} je zlonamerni program otkriven 2010. godine koji je naneo ogromnu štetu Iranskim nuklearnim elektranama. Pretpostavlja se da je delo Američkih i Izraelskih bezbednosnih agencija. Dizajniran je za napad na programabilne logičke kontrolere koji omogućavaju automatizaciju elektromehaničkih procesa, kao što je kontrola centrifugalnih mašina koje odvajajaju nuklearni materijal.

Koristeći četiri \textit{zero-day}\footnote{Zero-day propusti predstavljaju ranjivosti sistema za koje zna samo napadač.} propusta u operativnom sistemu \textit{Microsoft Windows}, virus je bio u mogućnosti da sebe iskopira na sve računare u lokalnoj mreži potpuno neprimetno. Kada se bi se našao na nekom od računara \textit{Stuxnet} je menjao kôd \textit{Step7} alata kompanije \textit{Siemens} za upravljanje kontrolerima i ubrzavao rad centrifugalnih mašina kako bi vremenom došlo do njihovog kvara. Korisnicima bi bili prikazani lažni podaci na osnovu kojih se ne bi moglo zaključiti da nešto nije u redu. \textit{Stuxnet} se sastoji od tri modula od kojih jedan izvodi glavni deo napada, drugi prebacuje virus na ostale računare, a treći sakriva sve zlonamerne procese i datoteke kako bi onemogućio otkrivanje. U sistem je ubačen preko zaraženog \textit{USB} uređaja.

Na samitu o računarskoj bezbednosti u Meksiko Sitiju 2015. godine laboratorija Kasperski (rus. \textit{Лаборатория Касперского}) je objavila otkriće postojanja \textit{Equation} grupe, koja je dobila ime zbog izuzetno sofisticiranog algoritma za šifrovanje koji je razvila. Izveštaji navode da grupa postoji od 2001. godine. Zbog veoma naprednih tehnika i visokog nivoa prikrivenosti, grupa se povezuje sa nacionalnom bezbednosnom agencijom (eng. \textit{National Security Agency, NSA}) Sjedinjenih Američkih Država. Otkriveno da su pre ili u isto vreme koristili iste \textit{zero-day} napade koji su korišćeni i u virusu \textit{Stuxnet} te se pretpostavljalo da su oni bili odgovorni za razvoj ovog virusa. Laboratorija Kasperski je grupu opisao kao daleko najsofisticiraniju koja je ikada otkrivena.

Pored virusa \textit{Stuxnet} povezuju se i sa zlonamernim programom \textit{Flame} otkrivenim 2012. godine koji predstavlja najkompleksniji virus ikada napravljen \cite{flame}. Sastoji se od deset grupa modula koji su prikazani u tabeli \ref{table:1}.

\begin{table}[h!]
\centering
\begin{tabular}{||c  c||} 
 \hline
 \textbf{Ime grupe} & \textbf{Namena} \\ 
 Flame & Moduli zaduženi za upad u sistem \\ 
 Boost & Moduli zaduženi za prikupljanje informacija \\
 Flask & Tip modula za napad \\
 Jimmy & Tip modula za napad \\
 Munch & Moduli za instalaciju i dalje širenje \\
 Snack & Moduli za lokalno širenje virusa \\
 Spotter & Moduli zaduženi za skeniranje sistema \\
 Transport & Moduli zaduženi za umnožavanje \\
 Euphoria & Slanje podataka ka ciljnim serverima \\
 Headache & Parametri i ostale karakteristike napada \\ [1ex] 
 \hline
\end{tabular}
\caption{Moduli zlonamernog programa Flame i njihova moguća namena}
\label{table:1}
\end{table}

Ovaj virus se koristio za špijunažu u zemljama bliskog istoka, pretežno Iranu. Program je mogao da se širi preko lokalnih mreža, kao i preko uređaja koji poseduju {\em Bluetooth}. Mogao je nezapaženo da snima zvuk i ekran, da registruje unos sa tastature i svu komunikaciju preko mreže. Informacije je slao serverima, koji su korišćeni za upravljanje i čekao dalja uputstva od njih. Posedovao je i ‘\textit{kill}’ komandu koja je brisala sve tragove postojanja virusa na računaru, a koja je poslata odmah pošto je virus prvi put otkriven. U izveštaju kompanije Kasperski se navodi da je \textit{Equation} grupa ta koja je pomogla napadačima dok je \textit{CrySyS Lab}\footnote{ Laboratorija za kriptografiju i bezbednost sistema, Univerzitet tehnologije i ekonomije u Budimpešti.} kao odgovorne za razvoj virusa \textit{Flame} označio bezbednosnu agenciju zemlje sa 'značajnim budžetom i kapacitetima' \cite{flame}.

\subsection{Umetnost obmane}
\label{deception}

Sa razvojem bezbednosnih tehnologija, postaje sve teže iskoristiti tehničke nedostatke sistema. Posledica toga je sve veće iskorišćavanje ljudskog faktora, što je često veoma lako. Vrsta napada, koji se svodi na prevaru žrtve da oda poverljive informacije ili napravi izmene u sistemu na kojem radi, a koje nisu u njenom interesu, naziva se društveni inženjering (eng.~{\em Social engineering}) \cite{deception}.

Napadi društvenih inženjera uspevaju kada ljudi nisu dovoljno upoznati sa dobrom praksom u bezbednosti. Uprkos tome što kompanija ili pojedinac, zarad svoje bezbednosti, kupi sve moguće alate i obuči svoje zaposlene o merama predostrožnosti, ona je i dalje ranjiva jer je ljudski faktor najslabija karika bezbednosnog sistema.

Osobe koje koriste nedostatke u ljudskoj prirodi, zarad ostvarenja svog cilja, poznate su kao društveni inženjeri (eng.~{\em Social engineers}). Društveni inženjeri umeju sa ljudima, šarmantni su i harizmatični - osobine koje omogućavaju pridobijanje poverenja i prisnosti sa žrtvom. Iskusni društveni inženjer, koristeći svoje umeće obmane, je u stanju da izvuče željenu informaciju ili inicira određenu radnju žrtve. Strategije za napad, koje oni koriste, sačinjene su od više različitih pristupa. Navedeni su neki od njih \cite{deception}.

\paragraph{Elementi strategije društvenih inženjera}
\begin{itemize}
\item \textbf{Skrivena vrednost informacija}

Društveni inženjeri često koriste informacije o firmi, koje su naizgled neškodljive, a mogu biti od velikog značaja za ostvarenje cilja. Poznavanje osnovnih pravila, načina ponašanja i izražavanja, procedura, često korišćenih termina i skraćenica unutar firme ili unutar određenog odeljenja firme, mogu biti presudan parametar napada (\ref{hack_za_ginisa}).

\item \textbf{Zloupotreba poverenja}

Poverenje je ključ prevare. Društveni inženjeri planiraju svoj napad vrlo temeljno, pokušavaju da predvide pitanja koja bi žrtva mogla postaviti. Pokušavaju da vode razgovor koji, za žrtvu, ne izlazi van okvira uobičajenog. Kada žrtve nemaju razloga za sumnju, napadačima je lako da pridobiju njihovo poverenje i, koristeći svoje veštine, ostvare svoje ciljeve.

\item \textbf{Pomoć neprijatelja}

Kada su ljudi uplašeni ili pod određenim pritiskom zbog nekog problema, skloni su da prihvate savete od osoba, koje ostavljaju utisak poverenja, a koje nikada nisu sreli. Napadač kroz pružanje pomoći žrtvi ostvaruje svoj cilj. On navodi žrtvu da instalira zlonamerni softver ili otkrije poverljive informacije, koje žrtva prepušta u nadi da će dovesti do rešenja problema. Početni problem može biti prouzrokovan od strane napadača (\ref{pomoc_neprijatelja}).

\item \textbf{Iskorišćavanje osećanja}

Iskusni društveni inženjeri su vešti u manipulisanju osećanjima žrtve. To čine koristeći psihološke okidače (eng.~{\em psychological triggers})\footnote{Automatski mehanizmi ljudske psihologije koji ih čine podložnim sugestiji.}, koji dovode do površnog pristupa žrtve problemu bez temeljne analize dostupnih informacija. Napadač izaziva saosećajnost žrtve izmišljajući problem, koji je žrtva prethodno i sama iskusila. Pored ovog pristupa, napadač može izazvati osećaj krivice kod žrtve ili koristiti zastrašivanje kao oružje.


\end{itemize}


\subsubsection{Najveća računarska prevara u istoriji}
\label{hack_za_ginisa}

Stenli Rifkin je, 1978. godine, radio za kompaniju zaduženu za razvoj sistema za bezbednost jedne od najvećih banaka u Los Anđelesu, pa je imao pristup poverljivim informacijama te banke. Radeći na odeljenju za prenos novca na daljinu (eng.~{\em wire room}), upoznao se sa procedurom transakcije. Službenici koji su imali ovlašćenje da nalože transfer, svakog jutra su dobijali kôd koji su u toku tog dana koristili u procesu autorizacije. Službenici bi najčešće napisali tu šifru na parčetu papira i ostavili je na sebi vidno mesto. 

Jednom je izvršavajući svoja zaduženja na odeljenju, uspeo da pročita aktuelni kôd sa parčeta papira jednog od službenika i da ga zapamti. Otišao je do govornice, odakle je pozvao jednog od zaposlenih u odeljenju za prenos novca i predstavio se kao član međunarodnog odeljenja banke. Posle uspešne autorizacije kôdom službenika, naložio je prenos deset miliona dolara na račun u Švajcarskoj, koji je prethodno otvorio. Postojao je deo koji je napadač propustio prilikom upoznavanja sa samom procedurom. Bio mu je potreban i broj međupartijskog poravnanja (eng.~{\em interoffice settlement number}). Nazvao je drugo odeljenje u banci, predstavio se kao zaposleni sa kojim je upravo razgovarao i tražio broj koji mu je nedostajao, uz obrazloženje da ga je zaboravio. Kada je saznao broj, ponovo je pozvao sobu za transfer novca na daljinu i uspešno završio transakciju. U Švajcarskoj je podigao novac kojim je kupio dijamante, koje je kasnije u kaišu prošvercovao nazad. Iako za ovaj poduhvat nije koristio računar, njegova prevara ušla je u Ginisovu knjigu rekorda u kategoriji 'najveća računarska prevara' \cite{deception}.

\subsubsection{Dozvoli mi da ti pomognem}
\label{pomoc_neprijatelja}

Napadač je nazvao zaposlenog u kompaniji, čije tajne podatke je želeo da sazna i predstavio se kao tehnička podrška. Obavestio ga je da su u narednom periodu mogući problemi na mreži i da u slučaju nekog kvara pozove prvo njega. Posle nekoliko dana je pozvao mrežni operativni centar kompanije i predstavio se kao zaposleni u kompaniji, koristeći podatke radnika, kojeg je prvobitno nazvao. Od operatera je tražio da isključe mrežu u njegovoj kancelariji, kako bi otklonio kvar. Sada je mreža, u kancelariji žrtve napada, bila isključena, pa nije mogao da preuzme potrebne datoteke sa servera, razmenjuje informacije sa kolegama, proveri elektronsku poštu ili koristi štampač.

Zaposleni je, upozoren da do ovoga može doći, pozvao broj koji je nekoliko dana pre toga dobio od napadača. Napadač se trudio da zvuči željno da pomogne kolegi u nevolji, ali i da ga ubedi da mu pružanjem te pomoći čini veliku uslugu. Nakon nekog vremena pozvao je mrežni operativni centar i zatražio uključenje mreže. Posle toga je ponovo pozvao zaposlenog i koristeći osećaj zahvalnosti koji je otklanjanjem problema izazvao, lako ga ubedio da instalira aplikaciju, koju mu je predstavio kao zaštitu od budućih problema sa mrežom. U pitanju je bio računarski virus - trojanski konj (\ref{Odisej}), koji je napadaču omogućio potpunu kontrolu nad žrtvinim računarom \cite{deception}.

\section{Zlonamerni softver}

Zlonamerni softver (eng.~{\em Malware}) predstavlja softver koji je dizajniran tako da nanese štetu ciljnom korisniku. U zavisnosti od namene, funkcije, i opasnosti postoji više vrsta zlonamernog softvera. Manje opasni programi mogu samo zauzeti procesorsko vreme ili mesto u memoriji i na taj način usporiti rad računara, dok opasniji mogu uništiti podatke na računaru, pa čak i preuzeti potpunu kontrolu nad njim. Na taj način računar postaje alat koji se može upotrebljavati za razne ilegalne aktivnosti poput krađe kreditnih kartica ili napada na druge računare u mreži \cite{ethics}.


\subsection{Virusi}
\label{sec:malware}

Računarski virus je program koji se ugrađuje u izvršni kôd drugih programa. Namena računarskog virusa je da zarazi loše obezbeđene delove sistema, preuzme kontrolu ili ukrade poverljive podatke. Načini na koje se virus širi su otvaranje dodatka u okviru elektronske pošte, poseta zaraženom sajtu, pokretanje sumnjive izvršne datoteke ili povezivanjem zaraženog prenosnog uređaja sa računarom. Računarski virus može raditi na dva načina. Može se aktivirati čim dospe u sistem ili naknadnom akcijom korisnika \cite{viruses_and_worms,computer_virus}.
\paragraph{Tipovi virusa}
\begin{itemize}
\item \textbf{Virus sektora za podizanje sistema}

Virus sektora za podizanje sistema (\textit{eng. Boot Sector Virus}) napada uređaje za skladištenje podataka, kao što su tvrdi (eng. \textit{hard drive}) ili \textit{floppy} diskovi, tako što izmeni njihovu glavnu sekciju za podizanje sistema. Kada se sistem jednom podigne sa zaraženim programom, virus se učita u radnu memoriju i dalje širi na sve \textit{floppy} diskove ubačene u računar. Kako se u moderne operativne sisteme ugrađuju čuvari te kritične sekcije (eng. \textit{safeguard}), a \textit{floppy} diskovi su izašli iz upotrebe, ovaj tip virusa je prevaziđen \cite{viruses_and_worms, computer_virus}.


\item \textbf{Virus direktnih akcija}

Virus direktnih akcija (\textit{eng. Direct Action Virus}) se aktivira čim dospe u sistem, ne sakriva se u memoriji i cilj mu je da se svakim izvršavanjem dodatno proširi kroz sistem. Virus inficira najpre tipove datoteka koje napadač odredi, a zatim i sistemske datoteke odgovorne za određene operacije prilikom podizanja sistema. Ovaj tip virusa nema za cilj brisanje datoteka ili smanjivanje performansi računara već prikupljanje i menjanje pristupa određenim podacima. Antivirusi veoma lako primete aktivnosti ovog tipa virusa. \cite{directaction}.


\item \textbf{Prikriveni virus}

Nasuprot virusu direktnih akcija prikriveni virus (\textit{eng. Resident Virus}) je specifičan po tome da ga je teško identifikovati i odstraniti. Virus se instalira na sistemu i krije u memoriji računara. Modularan je, pa je tako svaki deo virusa zadužen za različitu malicioznu akciju. Korišćenje ovakvog tipa virusa indukovano je poznavanjem masivnih propusta na samom sistemu \cite{computer_virus}.

\item \textbf{Polimorfni virus} (\textit{eng. Polymorphic Virus})

Praktično je imun na tradicionalne antivirus programe. Ovaj tip virusa menja svoj potpisni obrazac tj. svoj izvršni kôd svaki put kada se kopira. Ovo dovodi do otežanog lociranja virusa pomoću antivirus programa, kao i da se sam virus jako brzo širi kroz ceo sistem.

\item \textbf{Prepisujući virus} (\textit{eng. Overwrite Virus})

Odlikuje se time da briše sve datoteke koje su zaražene. Jedini način da se virus odstrani sa korisničkog računara je brisanje samih fajlova. 

\item \textbf{Virus za zauzimanje prostora - možda pre u fazonu Virus koji koristi prazan prostor, ovako je kao da zauzima memorijski prostor - Cane} (\textit{eng. Spacefiller})

Ubacuje se na prazna mesta u izvršnom kôdu i na taj način ne povećava veličinu zaraženog programa.

\end{itemize}


\subsection{Softver za špijuniranje}
\label{spyware}
Softver za špijuniranje (\textit{eng. Spyware}) je neželjeni softver koji upada u sistem  i krade poverljive informacije. Softver za špijuniranje sakuplja lične podatke korisnika i šalje ih raznim kompanijama koje se bave reklamama, obradom podataka i drugim korisnicima. Obično mu je cilj da dodje do lozinki, brojeva kreditnih kartica i podataka o bankovnim računima. 
To postiže tako što prati aktivnost korisnika na internetu, prati njegove informacije pri logovanju i špijunira osobu dok je na internetu. Neki tipovi softvera za špijuniranje mogu da instaliraju dodatni neželjeni softver na uređaj i čak da menjaju podešavanja, tako da je veoma bitno da se koriste bezbedne lozinke i da se uređaji redovno ažuriraju.\cite{spyware}

\paragraph{Tipovi softvera za špijuniranje}
\begin{itemize}
\item \textbf{Adware}

Koristi se za marketing tako što prati vasu istoriju pretraživača i skinute podatke sa interneta, sa 
namerom da predvidi kakvi bi proizvodi zainteresovali korisnika.
Može znatno da uspori računar.

\item \textbf{Trojanac}

Prerušava se u legalan softver. 
Može da ukrade lične podatke korisnika kao što su šifra ili broj kreditne kartice.

\item \textbf{Kolačići za praćenje}

Prate sve što korisnik radi na internetu(istorija pretraživanja, razmena podataka,...).

\item \textbf{System monitors}

Može da isprati sve sto korisnik radi na računaru,
registruje svaki pritisak dugmeta na tastaturi, pročitan email, razgovor na internetu, posećene sajtove i korišćene programe.
    
\end{itemize}   

\subsection{Mreža botova}

Mreža botova (eng.~{\em Botnet}) je kolekcija uređaja povezanih internetom koji su zaraženi i koje kontroliše zajednički tip malvera. Korisnici obično nisu svesni da im je sistem zaražen mrežom botova.

Zaražene uređaje daljinski kontrolišu ljudi, obično sajber kriminalci i koriste ih samo za specifične svrhe, tako da njihove zlonamerne akcije budu skrivene od korisnika. Mreže botova se obično koriste za slanje neželjene elektronske pošte, da prevare korisnike da posete sajtove koji generišu zlonamerni saobraćaj i druge.

Mreža botova traži i napada nedovoljno zaštićene uređaje, radije nego da cilja specifične kompanije ili pojedince. Cilj je da se mreža proširi na što više uređaja i da se njihova kompjuterska moć i resursi iskoriste za automatske radnje koje ostaju skrivene od korisnika uređaja.

Na primer, mreža koja zarazi neki računar zauzima internet pregledač da bi slala razne reklame. Da bi ostala skrivena, mreža ne zauzima ceo pretraživač nego samo jedan njegov mali deo, i šalje jedva primetni saobraćaj sa zauzetog uređaja ka tim reklamama.

Samostalan uređaj nije dovoljan, ali mreža koja broji milione uređaja može da proizvede ogroman saobraćaj reklamnog sadržaja i da ostane potpuno neprimećena.

\subsubsection{Arhitektura mreže botova}



\subsection{Trojanski konj}
\label{Odisej}

Trojanski konj (eng.~{\em Trojan horse}) slično kao i čuveni Trojanski konj iz Homerove Odiseje, predstavlja štetni program koji je maskiran u program koji izvršava korisne akcije na samom računaru. Kada se ovaj tip programa pokrene on izvršava neku akciju koja naizgled deluje kao korisna, ali u pozadini izvršava zlonameran softver bez znanja korisnika. \cite{trojanhorse}


Primer trojanskog konja predstavlja softver ~{\em Mocmex} otkriven 2008. godine, koji se nalazio na digitalnim ramovima za slike u Kini. Umesto samo učitavanja slika ovaj softver je, kada bi bio priključen na računar, prikriveno krao lične informacije o korisnicima, šifre za ??online?? igre i slično \cite{ethics}. 

\subsection{Računarski crvi}
Računarski crv (eng.~{\em Computer worm}) je maliciozni program koji se umnožava, a zatim širi među međusobno povezanim računarima. Širenje crva unutar mreže računara uslovljeno je propustima u toj mreži ili operativnom sistemu \cite{norton_worm, ethics}.

Računarski crv ne mora nužno biti maliciozan, ali uvek vrši neku vrstu ugrožavanja sistema makar kroz usporavanje protoka podataka kroz mrežu. Česta primena crva je da kroz veliko širenje omogući uslove za druge vrste napada. U tom slučaju crv najčešće instalira program pomoću kojeg se tim računarom može daljinski upravljati. \cite{ethics}

Ideja o ovakvom tipu softvera prvi put je predstavljena u naučno-fantastičnom romanu napisanom 1975. godine, autora Džona Brunera u kom se pominju crvi, programi koji se šire kroz mrežu futurističkih kompjutera. Nedugo zatim, 1979. godine, nekolicina zaposlenih u firmi ~{\em PARC } u Kaliforniji pravi i testira ovakav tip programa.
Prvi računarski crv je zapravo bio dobronameran. Korišćen je za brzo širenje korisnih informacija kroz mrežu \cite{internet_worm}. 


\subsubsection{Internet crv}

Termin Internet crv  (eng.~{\em Internet worm }) se najčešće poistovećuje sa terminom Morisov crv, odnosno programom Roberta Morisa koji je drugog novembra 1988. godine naneo višemilionsku štetu korisnicima interneta u Americi. Ovo je prvi napad korišćenjem ovakvog tipa programa i kao takav je dobio veliku pažnju medija i javnosti \cite{ethics}.

Internet crv je koristio greške u samom kôdu~{\em Unix} operativnog sistema u komandama~{\em fingerd} i~{\em sendmail}, kao i tehnike za otkrivanje jednostavnih šifara i brzo se proširio kroz mrežu na računare vojske, univerziteta pa i bolnica. Usled greške u samom kôdu crva, program se kopirao na iste računare po par stotina puta. Ovo je dovelo do toga da brojni računari otkažu usled prevelikog broja programa koji se izvršavaju na njima \cite{internet_worm}.


\subsubsection{Saser}
Lansiran 2004 godine, Saser (eng.~{\em Sasser}) je vrsta računarskog programa koji je izazvao velike troškove širom sveta i podigao svest javnosti o važnosti ažuriranja softvera. Saser je za širenje kroz skoro 18 miliona računara širom sveta iskoristio grešku u starijoj verziji~{\em Windows} operativnog sistema. Tako su računari koji nisu imali aktuelnu verziju operativnog sistema, među kojima su bili računari Austrijske železnice i Britanske obalske straže, postali neupotrebljivi \cite{ethics}.

\subsection{Ransomware}
https://www.us-cert.gov/Ransomware

\subsection{Načini pokretanja zlonamernog softvera}

CSS i ovaj drive-by nisu malver ili tip malvera nego tehnike pomoću kojih se malver bez znanja korisnika, a uz njegovu asistenciju (npr. posetio je sajt), instalira na sistem. Ne treba da budu dve posebne sekcije posebno sto sadržaj preti da nam pređe van prve strane, a verovatno imate da ubacite jos tipova zlonamernog softvera koliko se secam da je bilo u Mileninoj knjizi.

\subsubsection{Ukrštanje veb lokacija}

Koristeći tehniku ukrštanja veb lokacija (eng.~{\em Cross-Site scripting }) zlonamerni korisnik može postaviti kod unutar veb stranice, koristeći propuste u bezbednosti sajtova. Postavljen kod se izvršava u trenutku kada korisnik poseti datu stranicu i tada napadač najčešće dolazi do informacija o korisniku. \cite{ethics,xss}

Čest primer propusta na sajtovima je način smeštanja komenatara na serveru gde se komentar čuva u bazi podataka bez neke obrade ili provere a zatim prikazuje na klijentskoj strani veb aplikacije. Napadač postavljajući komentar u određenom formatu koji je ustvari kod može izmeniti stranicu tako da se prilikom posete korisnika veb aplikacije aktivira program koji šalje podatke korisnika napadaču. Ti podaci su najčešće kolačići ali neretko i unesene šifre i podaci o kreditnim karticama. \cite{xss}



\subsubsection{Usputno preuzimanje podataka }

Usputno preuzimanje podataka (eng.~{\em Drive by downloads}) predstavlja napad na računar korisnika zaraženog sajta na način da se 
na njegov računar kopiraju malicioziozni fajlovi. Napadač može zaraziti loše obezbeđen sajt tako da se na kompjuter svakog posetioca tog sajta kopira određen virus. Ovo kopiranje može se vršiti bez znanja korisnika ili pak preuzimanjem navodno obaveznog softvera iza kog se krije virus ili neka druga vrsta malicioznog softvera \cite{drivebydownloads, ethics}.

\section{Odbrana, zločin i kazna}

Pišem sutra ovo...

\subsection{Odbrana}

Imam jasnu sliku u glavi kako će da izgleda.

\subsection{Zločin i kazna}

Laku noć...
Aj laku noc :D


\section{Zaključak}
\label{sec:zakljucak}

'Sigurnosni sistem mora da pobedi svaki put, dok je napadaču dovoljno da pobedi samo jednom' \cite{intrusion}
Obezbeđenje je često iluzija sigurnosti. \cite{deception}

\addcontentsline{toc}{section}{Literatura}
\appendix
\bibliography{seminarski} 
\bibliographystyle{plain}

\appendix
\section{Dodatak}

\subsection{Hakovanje zarad uzbuđenja}
\label{mladi_hakeri}

Džonatan Džejms, u potpisu \textit{Comrade}, je najmlađa osoba ikada osuđena za hakovanje na teritoriji Sjedinjenih Američkih Država. Njegov prijatelj \textit{Ne0h} je već sa deset godina počeo da se bavi hakovanjem.  \textit{Comrade} i \textit{Ne0h} su se upoznali preko sajtova koji su predstavljali sobe u kojima su ljudi sličnih interesovanja mogli da razmenjuju informacije u tekstualnom formatu (eng. \textit{Internet Relay Chat, IRC}). Hakeri se često udružuju u grupe kako bi razmenjivali informacije i organizovali grupne napade.

Sredinom 1998. godine na jednom od tih sajtova \textit{Comrade} je stupio u kontakt sa Khalidom Ibrahimom, čovekom za koga se u zajednici pričalo da regrutuje ljude za upade na vladine sajtove i koji radi za, tada ne toliko poznatog teroristu, Bin Ladena. Khalid je stupio u kontakt i sa \textit{Ne0h}-om kome obećao 1000\$ ukoliko uspe da hakuje tehnički univerzitet u Kini. Ovaj zadatak, koji je bio samo test njegovih sposobnosti, je uspešno obavio.

Nakon nekoliko uspešnih probnih zadataka, od njega je zatraženo da hakuje avio kompaniju Boing (eng. \textit{Boeing}). \textit{Ne0h} je uspeo da provali širi obruč mreže i ostavi program koji mu je dao uvid u sve dolazne i odlazne pakete (eng. \textit{sniffer}). Na ovaj način uspeo je da sazna nekoliko korisničkih imena i lozinki koji su mu omogućili da uđe dublje u sistem. Uspeo je da dođe u posed i isporuči Khalidu šeme vrata na avionu Boing 747, šemu pilotske kabine kao i celog nosa aviona. Za \textit{Comrade}-a je imao samo jedan zadatak, da hakuje SIPRNET\footnote{Mreža koju je koristila vojska i druge vladine bezbednosne agencije za brz prenos naređenja na celoj teritoriji države.}. Prihvatio je ovaj zadatak, zbog izazova koji je predstavljao, i uspeo da na nekoliko računara na mreži ostavi program koji je pratio komunikaciju.

U isto vreme, u Indiji su teroristi oteli avion i počeli da kruže, praveći prinudna sletanja da napune rezervoare i izbace telo mladića koji se sa suprugom vraćao sa medenog meseca a kojeg su izboli jer nije hteo da stavi povez za oči. Sleteli su u Afganistan i tu čekali osam dana dok vlasti nisu pristale da oslobode trojicu ekstremista iz zatvora. Među njima i Sheikh Umer-a, koji će kasnije postati glavni finansijer Mohammed Atta-e vođe terorista koji su izveli napad 9.11.2001. Khalid je u razgovoru sa \textit{Comrade}-om rekao da je i on sam bio deo otmice. Kada je saznao za ovo, počeo je da uklanja tragove svojih aktivnosti povezanih sa njim.

Nekoliko dana posle njegovog upada u SIPRNET, njegovog oca je, na putu do posla, zaustavio FBI i predočio mu da imaju par pitanja za njegovog sina. Ispostavilo se da je računar na koji je uspeo da se infiltrira pripadao Nasi (eng. \textit{National Aeronautics and Space Administration, NASA}). Optužen je za tronedeljni zastoj u radu Nase kao i za presretanje više hiljada mailova ministarstva odbrane. Na jednom od njegovih hard diskova pronađen je program koji bi mu omogućio da kontroliše temperaturu i vlažnost vazduha na Internacionalnoj svemirskoj stanici. Iako star samo petnaest godina, osuđen je na šest meseci zatvora čime je pravosudni sistem poslao jasnu poruku maloletnim hakerima da im neće biti gledano kroz prste. 

U intervjuu 2002. godine jedan Indijski general je izneo informaciju da je u napadima koji su se desili godinu dana ranije umešan i Khalid Ibrahim koji je, kako je general izjavio, bio povezan sa hakerskim organizacijama i izrazio žaljenje što je baza ovog teroriste bio Nju Delhi. Jedna od tih hakerskih organizacija bila je i \textit{gLobaLheLL} čiji je vođa \textit{Zyklon} takođe imao kontakne sa Khalidom. Ta grupa uspela je krajem 1999. da upadne na sajt vojnog saveza \textit{NATO} kao i u mrežu Bele kuće i objavi zvučne i video zapise sa ličnog računara tadašnjeg predsednika Bila Klintona. U toku samog napada Khalid je stupio u kontakt sa \textit{Zyklon}-om kome se ovaj pohvalio da su upravo hakovali Belu kući. Nedugo zatim, iako su bili rani jutranji časovi i do tada nikog nije bilo na mreži, pojavio se sistem administrator i na mrežu instalirao program koji je pratio svu aktivnost i zabeležio adrese napadača.

Na saslušanju \textit{Zyklon} je imao uvid u dokumente u kojima je pisalo da su nadležni za napad saznali od FBI-ovog doušnika iz Nju Delhija. Ostalo je nepoznato da li je Khalid bio samo doušnik FBI-a ili dupli agent i pravi terorista kako je kasnije izjavio Indijski general i da li su informacije, koje su mu dala deca sa kojom je bio u kontaktu, na neki način pomogle u terorističkim napadima koji su sledili. Vlada je nakon svih tih dešavanja shvatila koliku pretnju predstavlja terorizam, kolike su zaista mogućnosti informatičkog ratovanja \ref{cyberwarfare} i koliko su njihovi sistemi bili izloženi \cite{intrusion}.

\end{document}
