

 % !TEX encoding = UTF-8 Unicode

\documentclass[a4paper]{report}

\usepackage[T2A]{fontenc} % enable Cyrillic fonts
\usepackage[utf8x,utf8]{inputenc} % make weird characters work
\usepackage[serbian]{babel}
%\usepackage[english,serbianc]{babel}
\usepackage{amssymb}

\usepackage{color}
\usepackage{url}
\usepackage[unicode]{hyperref}
\hypersetup{colorlinks,citecolor=green,filecolor=green,linkcolor=blue,urlcolor=blue}

\newcommand{\odgovor}[1]{\textcolor{blue}{#1}}

\begin{document}

\title{Hakovanje - večita igra nadmudrivanja\\ \small{Nikola Dimić, Đorđe Pantelić, Nikola Živković, Mladen Canović}}

\maketitle

\tableofcontents

\chapter{Recenzent \odgovor{--- ocena: 4} }


\section{O čemu rad govori?}
% Напишете један кратак пасус у којим ћете својим речима препричати суштину рада (и тиме показати да сте рад пажљиво прочитали и разумели). Обим од 200 до 400 карактера.
Opisano je poreklo nastanka hakovanja, kao i različiti razlozi koje ljudi imaju za bavljenje hakovanjem.
Navedeno je nekoliko primera velikih prevara koje su sprovedene uz manju ili veću pomoć računara.
Predstavljene su različite vrste zlonamernog softvera, među kojima su i virusi, računarski crvi i mreže botova.

\section{Krupne primedbe i sugestije}
% Напишете своја запажања и конструктивне идеје шта у раду недостаје и шта би требало да се промени-измени-дода-одузме да би рад био квалитетнији.
Nemam krupne primedbe.

\section{Sitne primedbe}
% Напишете своја запажања на тему штампарских-стилских-језичких грешки
Neke uočene pravopisne i štamparske greške:
\begin{itemize}
  \item američki (a ne Američki), slično i izraelski, iranski
  
  \odgovor{Sugestija je uvažena, prepravljene su greške u pasusu o zlonamernom programu \textit{Stuxnet}.}
  
  \item navodnici bi trebalo da se pišu ovako ,, `` (a ne ' ')
  
  \odgovor{Sugestija je uvažena, svi navodnici su prepravljeni i to u primerima:  \begin{itemize}
      \item značajnim budžetom i kapacitetima
      \item najveća računarska prevara
  \end{itemize}}
  
  \item koji (a ne koje) - Informatičko ratovanje predstavlja način na koje države...
  \odgovor{Sugestija je prihvaćena i rad je izmenjen na odgovarajući način.}

  
  \item jedno se viška - Kada se bi se našao na nekom od računara...
  
  \odgovor{Sugestija je prihvaćena i rad je izmenjen na odgovarajući način.}

  \item Bliski istok (a ne bliski istok)
  
  \odgovor{Sugestija je prihvaćena i rad je izmenjen na odgovarajući način.}

  \item zalažu (a ne žalažu)
  
  \odgovor{Sugestija je prihvaćena i rad je izmenjen na odgovarajući način.}

  \item disasembler (a ne disambler)
  
  \odgovor{Sugestija je prihvaćena i rad je izmenjen na odgovarajući način.}

\end{itemize}

Nekoliko stvari koje ,,ružno zvuče``:
\begin{itemize}
  \item tvrdi i flopi diskovi - bukvalan prevod
  
  \odgovor{Sugestija nije uvažena. Iako imaju čudan prizvuk, ovi pojmovi su opšte prihvaćeni u srpskoj literaturi.}

  
  \item zlokobne veb stranice - možda bolje zlonamerne 
  
  \odgovor{Sugestija je uvažena i rad je izmenjen u pasusu \textit{Mreža botova}.}

  \item Afganistan - u originalu jeste sa f, ali u srpskom je uobičajenije sa v
  
  \odgovor{Sugestija je uvažena i rad je izmenjen u dodatku.}

\end{itemize}

\section{Provera sadržajnosti i forme seminarskog rada}
% Oдговорите на следећа питања --- уз сваки одговор дати и образложење

\begin{enumerate}
\item Da li rad dobro odgovara na zadatu temu?\\
Da - temeljno je obrađena tema hakovanja i zlonamernog softvera.
\item Da li je nešto važno propušteno?\\
Ne - meni nije zasmetalo, ali u uvodu nedostaju reference, a na predavanjima je naglašeno da su na tom mestu najpotrebnije.

\odgovor{Sugestija nije uvažena. Definicija navedena u uvodu se bazira na jednoj referenci, koja je ispravno navedena. U ostatku uvoda se referiše na saržaj rada uz linkove ka poglavljima gde su navedene reference za ta poglavlja.}

\item Da li ima suštinskih grešaka i propusta?\\
Ne - ne primećujem velike greške.
\item Da li je naslov rada dobro izabran?\\
Da - odgovara temi i zanimljiv je, podstiče na dalje čitanje rada.
\item Da li sažetak sadrži prave podatke o radu?\\
Da - sažetak jasno prikazuje suštinu rada.
\item Da li je rad lak-težak za čitanje?\\
Da - većinom se rad lako čita, tekst je jasno napisan i nisu potrebni drugi izvori da bi se razumeo.
\item Da li je za razumevanje teksta potrebno predznanje i u kolikoj meri?\\
Ne - uglavnom nije potrebno neko posebno predznanje, osim uobičajenih informatičkih pojmova.
\item Da li je u radu navedena odgovarajuća literatura?\\
Da - navedena literatura je adekvatna i odgovara zahtevima (bar 1 knjiga, 1 članak, 1 veb strana).
Jedino nisam siguran za relevantnost blogova.

\odgovor{Deo blogova koji su sa veb stranica organizacija specijalizovnih za bezbednost na internetu kao što su \textit{Kasperski}, \textit{Norton} i \textit{Comodo} su relevantni. 
Kritičkim zaključivanjem autora blogovi o zakonima koji regulišu prava privatnosti na internetu su relevantni.
Autori tih blogova su profesori prava i studenti doktorskih studija informatike na svetski priznatim univerzitetima.}
\item Da li su u radu reference korektno navedene?\\
Da - reference su korektno navedene na adekvatnim mestima.
\item Da li je struktura rada adekvatna?\\
Da - uglavnom su ispoštovani zahtevi strukture rada.
Zamerka što se tiče strukture, u odnosu na zahteve sa predavanja, jesu pasusi koji se sastoje od samo jedne rečenice.

\odgovor{Zamerke su uvažene. Rad je izmenjen i pasusi su modifikovani tamo gde je po mišljenju autora bilo potrebno.}
\item Da li rad sadrži sve elemente propisane uslovom seminarskog rada (slike, tabele, broj strana...)?\\
Da - 1 slika i 1 tabela na 12 strana (uz jednu dozvoljenu dodatnu stranu za dodatak).
\item Da li su slike i tabele funkcionalne i adekvatne?\\
Da - slika jasno ilustruje ideju, a tabela adekvatno prikazuje podatke, i obe su referisane iz teksta.
\end{enumerate}

\section{Ocenite sebe}
% Napišite koliko ste upućeni u oblast koju recenzirate: 
% a) ekspert u datoj oblasti
% b) veoma upućeni u oblast
% c) srednje upućeni
% d) malo upućeni 
e) skoro neupućeni
% f) potpuno neupućeni
% Obrazložite svoju odluku


\chapter{Recenzent \odgovor{--- ocena: 3} }


\section{O čemu rad govori?}

\odgovor{Prema uslovima recenzije ovaj deo mora biti popunjen.}


% Напишете један кратак пасус у којим ћете својим речима препричати суштину рада (и тиме показати да сте рад пажљиво прочитали и разумели). Обим од 200 до 400 карактера.

\section{Krupne primedbe i sugestije}
Smatram da je rad vrlo dobro napisan, stoga nemam krupnih primedbi koje bih ovim putem izneo.
Što se tiče sugestija, ja bih spomenuo bar jedan noviji slučaj koji se dogodio na globalnom nivou u svetu hakovanja, poput hakovanja izbora za predsednika Amerike od strane Rusije.
Takođe, ima mnogo zanimljivih dogodovština u kojima su se najčešće ljudi posle ne tako malih hakerskih podviga prerano opustili i tako bivali uhvaćeni od strane nadležnih organa.

\odgovor{Sugestija je prihvaćena i dodata je sekcija A.2 u dotatku koja govori o navodnoj ruskoj umešanosti u američke predsedničke izbore.}


Isto tako, mislim da su autori mogli na bolji način da iskoriste prostor za sliku, na primer tako što bi izgenerisali neki grafik koji bi nam pokazao koji zlonamerni softveri su naneli više štete.

\odgovor{Iako su autori prvobitno imali ideju da vizuelno prikažu štetu nanetu pojedincima ili kompanijama, nisu pronađeni relevantni izvori podataka. Razlog za to je što ti relevatni izvori najčešće ne objavljuju osetljive podatke o svom poslovanju.}
% Напишете своја запажања и конструктивне идеје шта у раду недостаје и шта би требало да се промени-измени-дода-одузме да би рад био квалитетнији.

\section{Sitne primedbe}
Jedino što bih odvojio kao sitnu primedbu je deo rečenice u poglavlju 2.2 gde autori kažu da društveni inženjeri omogućavaju pridobijanje poverenja i prisnosti sa žrtvom - što ne zvuči prirodno, pa bih na tom mestu predložio da se rečenica preformuliše ili da se jednostavno izbaci deo ‘i prisnosti‘.

\odgovor{Sugestija je prihvaćena i rad je izmenjen na odgovarajući način.}

% Напишете своја запажања на тему штампарских-стилских-језичких грешки


\section{Provera sadržajnosti i forme seminarskog rada}
% Oдговорите на следећа питања --- уз сваки одговор дати и образложење

\begin{enumerate}
\item Da li rad dobro odgovara na zadatu temu?\\
Rad u potpunosti odgovara na zadatu temu, s obzirom na to da su pokriveni osnovni koncepti hakovanja i zlonamernog softvera - što kroz tehničke detalje, što kroz primere koji nam govore koliko ima domišljatih pojedinaca i kako razmišljaju ali i koliko je sigurnosni sistem napredovao u odnosu na njihove metode i zlonamerne softvere.
\item Da li je nešto važno propušteno?\\
Rekao bih da ništa važno nije propušteno.
\item Da li ima suštinskih grešaka i propusta?\\
Nema suštinskih grešaka niti propusta u radu.
\item Da li je naslov rada dobro izabran?\\
Naslov rada odgovara temi i izdvaja se iz mora monotonih radova, iako je mogao na neki način da se spomene i zlonamerni softver ali s obzirom na to da hakovanje samo po sebi obuhvata zlonamerni softver, naslov je dobro izabran.
\item Da li sažetak sadrži prave podatke o radu?\\
Sažetak sadrži prave podatke o radu, autori su na dovoljno apstraktan način uputili čitaoce kroz teme koje će se obrađivati i na koji način će im prići u radu. Kao i ciljeve koje rad ima pred sobom.
\item Da li je rad lak-težak za čitanje?\\
Rad je lak za čitanje i drži pažnju s obzirom na to da nije lako uvesti u temu nekoga ko nema puno dodirnih tačaka sa ovom oblašću.
\item Da li je za razumevanje teksta potrebno predznanje i u kolikoj meri?\\
Autori su na dobar način uveli čitaoce u rad i jasnom logičkom strukturom nadoknadili potrebno predznanje za praćenje narativa rada.
\item Da li je u radu navedena odgovarajuća literatura?\\
U radu je navedena odgovarajuća literatura koja je takođe i vrlo opširna što mi se jako dopada jer čini rad zanimljivijim i olakšava praćenje istog.
\item Da li su u radu reference korektno navedene?\\
Sve reference su korektno navedene.
\item Da li je struktura rada adekvatna?\\
Struktura je adekvatna iz razloga što u mnogome olakšava čitaocu da prati temu o kojoj se trenutno priča, zbog toga što nakon naslova podoblasti slede uvek njen uvod, zatim vrste, elementi ili priče koje nam približavaju tu podoblast.
\item Da li rad sadrži sve elemente propisane uslovom seminarskog rada (slike, tabele, broj strana...)?\\
Rad sadrži sve uslovom propisane elemente seminarskog rada.
\item Da li su slike i tabele funkcionalne i adekvatne?\\
Slika i tabela su funkcionalne i adekvatne i pomažu u preglednosti i praćenju rada.
\end{enumerate}

\section{Ocenite sebe}
Rekao bih da sam veoma upućen u oblast koju recenziram. Ne samo iz razloga što mi je zanimljiva i što aktivno pratim šta se dešava kako u kompjuterskom tako i u svetu hakovanja, već i zato što sam kao dete eksperimentisao i istraživao u velikim merama na zadatu temu.

% Napišite koliko ste upućeni u oblast koju recenzirate: 
% a) ekspert u datoj oblasti
% b) veoma upućeni u oblast
% c) srednje upućeni
% d) malo upućeni 
% e) skoro neupućeni
% f) potpuno neupućeni
% Obrazložite svoju odluku


\chapter{Recenzent \odgovor{--- ocena: 4} }


\section{O čemu rad govori?}
% Напишете један кратак пасус у којим ћете својим речима препричати суштину рада (и тиме показати да сте рад пажљиво прочитали и разумели). Обим од 200 до 400 карактера.
Tema rada je hakovanje - načini za neovlašćen pristup i zloupotrebu računarskih sistema. Posle uvodnog dela, u poglavlju 2 su opisani neki konkretni računarski napadi i zlonamerni softver (kao što je \em{Stuxnet}\em), kao i neke prevare vezane za računarske sisteme (koristeći takozvani \em{društveni inženjering}\em). U poglavlju 3 je dat pregled različitih vrsta zlonamernog softvera (eng. \em{malware}\em).

\section{Krupne primedbe i sugestije}
% Напишете своја запажања и конструктивне идеје шта у раду недостаје и шта би требало да се промени-измени-дода-одузме да би рад био квалитетнији.
Rad detaljno pokriva temu (koliko je to moguće, sa obzirom na obim) i nema značajnih nedostataka. Bilo bi dobro da se pri kraju rada (verovatno u zaključku) osvrne malo detaljnije na društvene aspekte hakovanja, tj. na odgovornost koja se spominje u sažetku.

\odgovor{Sugestija je prihvaćena. Dodata je sekcija 2.2.3 koja govori o društvenim aspektima hakovanja i odgovornosti modernog društva.}

Jedina kritika se tiče jednog citata: časopis \em{International Journal of Computer Science and Information Technologies }\em ([8] na listi literature) se nalazi na listama sumnjivih časopisa (\href{https://predatoryjournals.com/journals/}{1} i \href{https://beallslist.weebly.com/standalone-journals.html}{2}), pa bi trebalo da se zameni nekim drugim. Informacije u pitanju (uvodni deo drugog poglavlja) nisu sporne i svakako mogu da se nađu negde drugde.

\odgovor{Sugestija je uvažena. Sumnjiv izvor je zamenjen relevantnim. Navedena je referenca na rad \textit{Ninja Hacking: Unconventional Penetration Testing Tactics and Techniques}, autora Tomasa Vilhelma i Džejsona Andresa.} 

\section{Sitne primedbe}
% Напишете своја запажања на тему штампарских-стилских-језичких грешки
Postoji nekolicina malih isravki: U 2.1.1, 2. pasus treba ``disasembler'' umesto ``disambler''; u 2.1.2, 4. pasus ``nacionalna bezbednosna agencija'' treba da počinje velikim slovom; u 2.1.2, poslednji pasus ne treba zarez iza ``serverima'' (ili treba dodati zarez iza ``upravljanje'').

\odgovor{Sve navedene sugestije su prihvaćene i rad je ispravljen u skladu s njima.}


\section{Provera sadržajnosti i forme seminarskog rada}
% Oдговорите на следећа питања --- уз сваки одговор дати и образложење

\begin{enumerate}
\item Da li rad dobro odgovara na zadatu temu?\\
Da, rad daje temeljan pregled osnovnih principa i tehnika hakovanja.
\item Da li je nešto važno propušteno?\\
Nema značajnih propusta.
\item Da li ima suštinskih grešaka i propusta?\\
Ne, informacije su korektne i potkrepljene literaturom.
\item Da li je naslov rada dobro izabran?\\
Da, naslov je interesantan i dobro opisuje temu.
\item Da li sažetak sadrži prave podatke o radu?\\
Sažetak tačno opisuje sadržaj rada, uz primedbu da se ``odgovornost, koja je stavljena pred moderno društvo'' može malo detaljnije opisati.

\odgovor{Sugestija je uvažena, dodata je sekcija 2.2.3 u kojoj je detaljnije objašnjeno na koji način moderno društvo treba da odgovori na prethodno pomenute izazove.}

\item Da li je rad lak-težak za čitanje?\\
Rad je lak za čitanje, budući da je tema interesantna i da rad ne sadrži mnogo tehničkih detalja.
\item Da li je za razumevanje teksta potrebno predznanje i u kolikoj meri?\\
Nije potrebno ništa, sem eventualno osnovne ``računarske pismenosti''.
\item Da li je u radu navedena odgovarajuća literatura?\\
Jeste, uz jedan ``sumnjiv'' časopis, kao što je navedeno.

\odgovor{Sugestija je uvažena i rad je korigovan kao što je navedeno ranije u tektsu.}

\item Da li su u radu reference korektno navedene?\\
Da, svaki pasus sadrži barem jednu referencu koja ukazuje na izvor informacija.
\item Da li je struktura rada adekvatna?\\
U radu se uglavnom ređaju primeri ili vrste softvera, što je adekvatno sa obzirom na temu.
\item Da li rad sadrži sve elemente propisane uslovom seminarskog rada (slike, tabele, broj strana...)?\\
Da, rad ima 12 strana (bez dodatka), tabelu i sliku.
\item Da li su slike i tabele funkcionalne i adekvatne?\\
Da, funkcionalne su i njihov sadržaj je u skladu sa odgovarajućim delom rada.
\end{enumerate}

\section{Ocenite sebe}
% Napišite koliko ste upućeni u oblast koju recenzirate: 
% a) ekspert u datoj oblasti
% b) veoma upućeni u oblast
% c) srednje upućeni
% d) malo upućeni 
% e) skoro neupućeni
% f) potpuno neupućeni
% Obrazložite svoju odluku
Ocenio bih se kao ``srednje upućen'' u ovu oblast: većina činjenica i termina u radu mi je prethodno bila poznata, ali ne svi detalji (niti mnogo poznajem tehničke detalje hakovanja).



\chapter{Dodatne izmene}
%Ovde navedite ukoliko ima izmena koje ste uradili a koje vam recenzenti nisu tražili. 

\odgovor{Izmenjen je termin socijalni inženjering u socijalno inženjerstvo kako je više u duhu srpskog jezika.}

\end{document}
